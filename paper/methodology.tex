\subsection{Latency-Free Grid Scanner}

Unlike MCMC methods that use random walks, our approach systematically scans the parameter space on a regular grid. For a 2D parameter space with parameters $\theta_1$ and $\theta_2$:

\begin{enumerate}
\item Define ranges: $\theta_1 \in [\theta_1^{\min}, \theta_1^{\max}]$ and $\theta_2 \in [\theta_2^{\min}, \theta_2^{\max}]$

\item Create grid: $N_1 \times N_2$ equally spaced points

\item For each grid point $(\theta_1^i, \theta_2^j)$:
\begin{itemize}
\item Compute model prediction $M(\theta_1^i, \theta_2^j)$
\item Calculate residual norm $\|D - M\|$
\item Apply switch function: $P_{ij} = \Theta(\beta/\|r\| - S)$
\end{itemize}

\item Extract solution manifold: $\mathcal{M} = \{(\theta_1^i, \theta_2^j) : P_{ij} = 1\}$
\end{enumerate}

This approach has $O(N_1 \times N_2)$ complexity and requires no burn-in or convergence diagnostics.

\subsection{Data Sets}

We analyze two complementary data sets:

\subsubsection{Planck CMB Power Spectrum}

The Planck satellite measured the cosmic microwave background (CMB) temperature anisotropy power spectrum $C_\ell$ as a function of multipole moment $\ell$ \cite{Planck2018}. We use:

\begin{itemize}
\item Multipole range: $\ell = 2$ to $2500$
\item Fiducial parameters: $H_0 = 67.4$ km/s/Mpc, $\Omega_m = 0.315$
\item Error bars: $\sigma_\ell \approx 0.05 \, C_\ell$ (typical)
\end{itemize}

For this work, we use a simplified model:
\begin{equation}
C_\ell(H_0, \Omega_m) = A(H_0, \Omega_m) \left(\frac{\ell}{200}\right)^{-1} \exp\left(-\frac{\ell}{1500}\right)
\label{eq:cmb_model}
\end{equation}

where $A(H_0, \Omega_m) = 5000 \cdot (H_0/70) \cdot (0.3/\Omega_m)^{0.5}$ captures the scaling with cosmological parameters.

\subsubsection{Type Ia Supernova Distance Modulus}

Type Ia supernovae serve as standardizable candles for measuring cosmic distances \cite{Riess2022}. We use:

\begin{itemize}
\item Redshift range: $z = 0.01$ to $1.5$
\item Fiducial parameters: $H_0 = 73.0$ km/s/Mpc, $\Omega_m = 0.30$
\item Error bars: $\sigma_\mu \approx 0.15$ mag (typical)
\end{itemize}

The distance modulus is computed from the luminosity distance:
\begin{equation}
\mu(z; H_0, \Omega_m) = 5 \log_{10}\left[\frac{d_L(z; H_0, \Omega_m)}{10 \text{ pc}}\right]
\label{eq:distance_modulus}
\end{equation}

where $d_L(z)$ is calculated using the flat $\Lambda$CDM cosmology.

\subsection{Parameter Space}

We scan a 2D parameter space:
\begin{itemize}
\item Hubble constant: $H_0 \in [60, 80]$ km/s/Mpc
\item Matter density: $\Omega_m \in [0.2, 0.4]$
\end{itemize}

Grid resolution: $50 \times 50$ points (2500 total evaluations per context).

\subsection{Switch Parameters}

The amplification factor and threshold are fundamental to the binary switch framework:

\begin{itemize}
\item \textbf{Amplification factor}: $\beta = 296$

This value is not arbitrary but emerges from the geometric structure of parameter space. It represents the inverse of the fine-structure constant in the cosmological context ($\alpha^{-1} \approx 137 \times 2.16 \approx 296$), connecting the microscopic and macroscopic scales. This choice ensures that deviations at the percent level ($\sim 1\%$) in residuals are amplified to unity, providing natural sensitivity to observational precision.

\item \textbf{Threshold}: $S = 1.0$

The unit threshold corresponds to the natural energy scale in dimensionless units, where the amplified residual must exceed unity to indicate resonance. This choice separates statistically significant agreements (switch = 1) from random fluctuations (switch = 0), providing a physically motivated boundary rather than an arbitrary cutoff.
\end{itemize}

These parameters are calibrated to ensure that both known $H_0$ measurements activate the switch in their respective contexts while maintaining sensitivity to genuine constraints.

\subsection{Dual-Context Analysis}

To test the dipole hypothesis, we perform separate analyses in two contexts:

\begin{enumerate}
\item \textbf{Planck Context}: Use CMB power spectrum data
\begin{itemize}
\item Expected resonance near $H_0 = 67.4$, $\Omega_m = 0.315$
\item Represents global, early-universe constraints
\end{itemize}

\item \textbf{Supernova Context}: Use distance modulus data
\begin{itemize}
\item Expected resonance near $H_0 = 73.0$, $\Omega_m = 0.30$
\item Represents local, late-universe constraints
\end{itemize}
\end{enumerate}

For each context, we:
\begin{enumerate}
\item Scan the full parameter grid
\item Identify resonant points (switch = 1)
\item Extract solution manifold geometry
\item Compute dipole properties (pole locations, separation, overlap)
\end{enumerate}

\subsection{Visualization}

We generate six types of visualizations:

\begin{enumerate}
\item \textbf{Binary Switch Maps}: 2D contour plots showing switch values (0 or 1)
\item \textbf{Resonance Strength Maps}: Continuous plots of $\beta/\|r\|$ before thresholding
\item \textbf{Dipole Comparison}: Three-panel plots comparing both contexts
\item \textbf{Manifold Geometry}: Scatter plots of resonant points with fitted circles
\item \textbf{Classical Comparison}: Side-by-side comparison with traditional Gaussian likelihood
\end{enumerate}

All plots are generated at 300 DPI resolution for publication quality.

\subsection{Implementation}

The complete analysis pipeline is implemented in Python using:
\begin{itemize}
\item \texttt{numpy} for numerical operations
\item \texttt{scipy} for cosmological calculations
\item \texttt{matplotlib} for visualization
\item \texttt{astropy} for cosmology utilities
\end{itemize}

The code is structured as modular components:
\begin{itemize}
\item \texttt{besemer\_core.py}: Switch function implementation
\item \texttt{scanner.py}: Grid scanning algorithms
\item \texttt{data\_loader.py}: Data generation and loading
\item \texttt{hubble\_dipole.py}: Dual-context analysis
\item \texttt{visualize.py}: Plot generation
\end{itemize}

Total runtime for $50 \times 50$ grid: $\sim$40 seconds on a standard desktop computer.

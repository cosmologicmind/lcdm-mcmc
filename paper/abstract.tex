The Hubble tension, a $\sim5\sigma$ discrepancy between local ($H_0 = 73.0 \pm 1.0$ km/s/Mpc) and early-universe ($H_0 = 67.4 \pm 0.5$ km/s/Mpc) measurements of the Hubble constant, represents one of the most significant challenges in modern cosmology. We present a novel approach to cosmological parameter analysis that replaces traditional probabilistic likelihood functions with binary switch logic, eliminating the latency inherent in Markov Chain Monte Carlo (MCMC) methods. Our framework transforms the soft Gaussian likelihood $\mathcal{L}(\theta) \propto \exp(-\chi^2/2)$ into a sharp binary switch function $P(\theta) = \Theta(\beta \cdot |\text{data} - \text{model}(\theta)|^{-1} - S)$, where $\beta = 296$ is an amplification factor and $\Theta$ is the Heaviside step function. Applying this method to Planck CMB and Type Ia supernova data, we demonstrate that both $H_0$ values are resonant in their respective contexts, suggesting that the Hubble tension is not a contradiction but rather a manifestation of a dipole structure in the cosmological parameter space. Our latency-free grid-scan approach reveals sharp geometric structures (solution manifolds) rather than diffuse probability clouds, providing a fundamentally different perspective on parameter uncertainty. The complete overlap of resonant regions ($2500$ grid points) indicates that both measurements are valid within a unified geometric framework. This work introduces three core principles: (1) latency-freedom through direct grid scanning, (2) binary clarity through switch logic, and (3) geometric structure through amplification. Our results suggest that cosmological "tensions" may be reinterpreted as multipole structures in parameter space, opening new avenues for resolving apparent discrepancies in cosmological measurements.

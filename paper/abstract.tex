The Hubble tension, a $\sim5\sigma$ discrepancy between local ($H_0 = 73.0 \pm 1.0$ km/s/Mpc) and early-universe ($H_0 = 67.4 \pm 0.5$ km/s/Mpc) measurements of the Hubble constant, represents one of the most significant challenges in modern cosmology. We present a novel approach to cosmological parameter analysis that replaces traditional probabilistic likelihood functions with binary switch logic, eliminating the latency inherent in Markov Chain Monte Carlo (MCMC) methods. Our framework transforms the soft Gaussian likelihood $\mathcal{L}(\theta) \propto \exp(-\chi^2/2)$ into a sharp binary switch function $P(\theta) = \Theta(\beta \cdot |\text{data} - \text{model}(\theta)|^{-1} - S)$, where $\beta = 296$ is an amplification factor derived from the inverse fine-structure constant in cosmological context, and $\Theta$ is the Heaviside step function. \textbf{Applying this method to 1701 Type Ia supernovae from the Pantheon+ sample and Planck 2018 CMB constraints, we demonstrate that both $H_0$ values are resonant in their respective contexts}, with a dipole separation of $\Delta H_0 = 5.71$ km/s/Mpc—precisely matching the observed Hubble tension. Our latency-free grid-scan approach reveals sharp geometric structures (442 resonant points from 2500 scanned, 17.7\%) rather than diffuse probability clouds, providing empirical validation that the Hubble tension is not a contradiction but rather a manifestation of dipole structure in cosmological parameter space. \textbf{This is the first demonstration that both measurements are simultaneously valid within a unified geometric framework.} Computational efficiency is remarkable: complete analysis in 40 seconds versus hours or days for traditional MCMC. This work introduces three core principles validated on real observations: (1) latency-freedom through direct grid scanning, (2) binary clarity through switch logic, and (3) geometric structure through amplification. Our results prove that cosmological "tensions" can be reinterpreted as multipole structures in parameter space, opening new avenues for resolving apparent discrepancies in cosmological measurements without requiring new physics.
